%Copyright 2014 Jean-Philippe Eisenbarth
%This program is free software: you can 
%redistribute it and/or modify it under the terms of the GNU General Public 
%License as published by the Free Software Foundation, either version 3 of the 
%License, or (at your option) any later version.
%This program is distributed in the hope that it will be useful,but WITHOUT ANY 
%WARRANTY; without even the implied warranty of MERCHANTABILITY or FITNESS FOR A 
%PARTICULAR PURPOSE. See the GNU General Public License for more details.
%You should have received a copy of the GNU General Public License along with 
%this program.  If not, see <http://www.gnu.org/licenses/>.


%Based on the code of Yiannis Lazarides
%http://tex.stackexchange.com/questions/42602/software-requirements-specification-with-latex
%http://tex.stackexchange.com/users/963/yiannis-lazarides
%Also based on the template of Karl E. Wiegers
%http://www.se.rit.edu/~emad/teaching/slides/srs_template_sep14.pdf
%http://karlwiegers.com
\documentclass{scrreprt}
\usepackage{listings}
\usepackage{underscore}
\usepackage{epstopdf}
\usepackage{booktabs}
\usepackage{xcolor}
\usepackage[bookmarks=true]{hyperref}
\usepackage[utf8]{inputenc}
\usepackage[english]{babel}
\hypersetup{
    bookmarks=false,    % show bookmarks bar?
    pdftitle={Software Requirement Specification for Recipe Buddy},    % title
    pdfauthor={Matthew Sprague, Brian Williams, Joseph Morrison, Jeffrey Rescignano},                     % author
    pdfsubject={Software Requirement Specification},                        % subject of the document
    pdfkeywords={Recipe, Buddy, Software, Requirement, Specification}, % list of keywords
    colorlinks=true,       % false: boxed links; true: colored links
    linkcolor=blue,       % color of internal links
    citecolor=black,       % color of links to bibliography
    filecolor=black,        % color of file links
    urlcolor=purple,        % color of external links
    linktoc=page            % only page is linked
}%
\def\myversion{0.1}
%\title
\usepackage{hyperref}
\begin{document}

\begin{flushright}
    \rule{16cm}{5pt}\vskip1cm
    \begin{bfseries}
        \Huge{SOFTWARE REQUIREMENTS\\ SPECIFICATION}\\
        \vspace{1.85cm}
        for\\
        \vspace{1.85cm}
        $Recipe Buddy$\\
        \vspace{1.8cm}
        \LARGE{Version \myversion}\\
        \vspace{1.8cm}
        Prepared by $ $Matthew Sprague, Brian Williams, Joseph Morrison, Jeffrey Rescignano$ $\\
        \vspace{1.8cm}
        $ $Certified Data Boys$ $\\
        \vspace{1.8cm}
        \today\\
    \end{bfseries}
\end{flushright}

\tableofcontents


\chapter*{Revision History}\label{revisions}

\begin{center}
    \begin{tabular}{c c c c}
        \midrule
        Name & Date        & Reason For Changes & Version \\
        \midrule
        All team members 
             & Sep 27 2021 & Initial doc        & 0.1     \\
        \midrule
    \end{tabular}
\end{center}

\chapter{Introduction}

\section{Purpose}
$ $Recipe Buddy is a modern recipe sharing website intended to facilitate connections between our users with food and culinary arts.
It creates a personalized experience for the user, showcases popular and trending recipes, and creates a community through various recipe sharing services.$ $

\section{Document Conventions}
$ $The SRS document has the following conventions:$ $
\subsection{Lists}
$ $Lists and sublists should be formatted as such:
\begin{itemize}
    \item This is an entry on a list!
    \item This is another entry on a list.
          \begin{itemize}
              \item This is an entry on a nested list!
              \item This is another entry on a nested list.
                    \begin{itemize}
                        \item And this is an item on a doubly nested list.
                    \end{itemize}
          \end{itemize}
\end{itemize}
Lists should never be nested more than twice.
The list in this section shows how far we allow our lists to go.
Anything more than that is not allowed.

\subsection{Revisions}
After a new revision, it is important to log your progress.
Please update this document's version number and update this document's Revision History table.

\subsection{Styling}
When updating this document, adhere to the following style guidelines:
\begin{itemize}
    \item The first paragraph of each section or subsection does not need to be indented.
          Subsequent paragraphs should be indented.
          In LaTeX, that is done by default by typing two blank spaces ``\ \ " after a paragraph.
    \item This document should be typed using the default LaTeX font and styling.
    \item When writing numbers within a paragraph:
          \begin{itemize}
              \item Whole numbers between zero and fifteen should be typed alphaetically.
              \item Whole numbers equal to or above 16 should be typed numerically.
              \item Decimal numbers, such as 3.14, should be typed numerically.
              \item Fractions should be typed in the form {$\frac{1}{2}$}.
                    In LaTeX, this can be done using the following code:
                    \\ \texttt{ \{\$ \textbackslash frac\{1\}\{2\} \$\} }
          \end{itemize}
\end{itemize}

\subsection{Code}
When writing code within a paragraph:
\begin{itemize}
    \item For short snippets, it is okay to use inline code. For example: \\
          The Python command \texttt{ print( ... ) } writes to standard output. \\
          In LaTeX, this is done using the \texttt{ \textbackslash texttt\{ \} } command in-line with the rest of a paragraph.
    \item For longer snippets, please give the code its own block. For example: \\
          The following Python function will calculate the area of a circle: \\
          \texttt{ import math \\
              \\
              // This function takes in the radius of a circle \\
              // and returns the area of that circle. \\
              def area\_circle(radius): \\
              \\
              r\_squared = radius ** 2 \\
              return math.pi * r\_squared \\
          }
    \item Please remember that this is a requirements document, and not the system's documentation!
          Remember to only use code in this document when it is needed!
          That is,
          \begin{itemize}
              \item Write code that clearly illustrates what you are doing.
              \item Write code that is concise.
              \item Explain what your code is doing such that a reader with minimal technical skill can understand what the code is doing.
              \item If the code takes too long to explain, save it for the system's documentation.
              \item Try not to take up more than half of a page with a block of code.
              \item Never take up more than an entire page with a block of code.
              \item Aim to have minimal overlap between the code in this document and the code in our system's documentation.
              \item The code in this document must convey meaningful information to all readers.
                    It should be useful for developers, as well as non-developers.
          \end{itemize}
\end{itemize}
$ $

\section{Intended Audience and Reading Suggestions}
$ $This document is aimed at a wide audience:
\begin{itemize}
    \item Recipe Buddy developers should be able to go over this document to understand how the system functions.
          Developers should have the ability to use this in conjunction with the system's documentation to gain a comprehensive understanding of the system's underlying features and overall design.
          Developers should also be able to use this document to understand who to design this system for and why the system is designed the way it is.
          That way, with each new proposed update to Recipe Buddy, developers can assess if it would align with the system's design philosphy.
    \item Recipe Buddy designers and clients should be able to go over this document to understand how the system is designed.
          With any proposed updates or revisions to the system, designers should have the ability to refer to this document and ensure such changes align with Recipe Buddy's design philosophy.
\end{itemize}
$ $

\section{Project Scope}
Recipe Buddy is a tool that allows registered users to organize their own recipies, view recipies of other registered users, and keep a catalog of food, kitchenware and applicances that they own. Recipe Buddy can also track dietary restrictions to offer a better user experince.
$ $Recipe Buddy organizes recipies by:
\begin{itemize}
    \item Allowing users to create a recipe.
    \item Allowing users to modify their recipies.
    \item Allowing users to delete their recipies.
    \item Allowing users to set the visibility of their recipies (i.e.\ public or private)
    \item Allowing users to share private recipies with specific users.
    \item Allowing users to offer suggestions to other's recipies.
\end{itemize}
$ $
$ $Recipe Buddy keeps a catalog of food and kitchenware items by:
\begin{itemize}
    \item Allowing for a detailed list of food items available in a users panty.
    \item Allowing for a user to add all items from a particular recipe into their panty.
    \item Allowing for users to keep a detailed list of kitchenware that they own.
    \item Allowing users to edit or modify these lists at any time from the user menu.
\end{itemize}
$ $
$ $Recipe Buddy tracks dietary restrictions by:
\begin{itemize}
    \item Allowing users to keep a detailed list of specific ingredients marked as allergens.
    \item Allowing users to keep a detailed list of restricted categories. (i.e.\ dairy for lactose intolerant or bread for gluten-free).
\end{itemize}
$ $

\section{References}
$<$List any other documents or Web addresses to which this SRS refers. These may 
include user interface style guides, contracts, standards, system requirements 
specifications, use case documents, or a vision and scope document. Provide 
enough information so that the reader could access a copy of each reference, 
including title, author, version number, date, and source or location.$>$


\chapter{Overall Description}

\section{Product Perspective}
Recipe Buddy is a multi-functional tool for organizing recipes, pantry items and dietary restrictions.
Although several recipe websites already exist, Recipe Buddy offers many powerful customizations and search features that seperate it from the competition.

\section{Product Functions}
$ $The Recipe Buddy program offers:
\begin{itemize}
    \item A limited publicly-accessible database of Recipes.
    \item User registration to access full features of the website.
    \item The organization of recipies by:
          \begin{itemize}
              \item Allowing users to create a recipe.
              \item Allowing users to modify their recipies.
              \item Allowing users to delete their recipies.
              \item Allowing users to set the visibility of their recipies (i.e.\ public or private)
              \item Allowing users to share private recipies with specific users.
              \item Allowing users to offer suggestions to other's recipies.
          \end{itemize}
    \item The organization of panty items and cookware by:
          \begin{itemize}
              \item Allowing for a detailed list of food items available in a users panty.
              \item Allowing for a user to add all items from a particular recipe into their panty.
              \item Allowing for users to keep a detailed list of kitchenware that they own.
              \item Allowing users to edit or modify these lists at any time from the user menu.
          \end{itemize}
    \item The tracking of dietary restrictions by:
          \begin{itemize}
              \item Allowing users to keep a detailed list of specific ingredients marked as allergens.
              \item Allowing users to keep a detailed list of restricted categories. (i.e.\ dairy for lactose intolerant or bread for gluten-free).
          \end{itemize}
    \item Powerful search features to sort and filter:
          \begin{itemize}
              \item By pantry and cookware items already available to the user.
              \item By user-specified dietary restrictions.
              \item By popularity, rating or difficulty.
          \end{itemize}
    \item Social features such as:
          \begin{itemize}
              \item Private sharing of a recipe.
          \end{itemize}
\end{itemize}
$ $

\section{User Classes and Characteristics}
$<$Identify the various user classes that you anticipate will use this product.  
User classes may be differentiated based on frequency of use, subset of product 
functions used, technical expertise, security or privilege levels, educational 
level, or experience. Describe the pertinent characteristics of each user class.  
Certain requirements may pertain only to certain user classes. Distinguish the 
most important user classes for this product from those who are less important 
to satisfy.$>$

\section{Operating Environment}
Although Recipe Buddy is expected to work on modern browsers, The Recipe Buddy client will only be tested extensively to ensure proper functionality across the top three major web browsers by marketshare including:
\begin{itemize}
    \item Google Chrome
    \item Apple Safari
    \item Mozilla Firefox
\end{itemize}

\section{Design and Implementation Constraints}
$<$Describe any items or issues that will limit the options available to the 
developers. These might include: corporate or regulatory policies; hardware 
limitations (timing requirements, memory requirements); interfaces to other 
applications; specific technologies, tools, and databases to be used; parallel 
operations; language requirements; communications protocols; security 
considerations; design conventions or programming standards (for example, if the 
customer’s organization will be responsible for maintaining the delivered 
software).$>$

\section{User Documentation}
Recipe Buddy will feature a web-based video tutorial that will help wesbite users and visitors explore Recipe Buddy features.

\section{Assumptions and Dependencies}

$<$List any assumed factors (as opposed to known facts) that could affect the 
requirements stated in the SRS.\@ These could include third-party or commercial 
components that you plan to use, issues around the development or operating 
environment, or constraints. The project could be affected if these assumptions 
are incorrect, are not shared, or change. Also identify any dependencies the 
project has on external factors, such as software components that you intend to 
reuse from another project, unless they are already documented elsewhere (for 
example, in the vision and scope document or the project plan).$>$


\chapter{External Interface Requirements}

\section{User Interfaces}
$<$Describe the logical characteristics of each interface between the software 
product and the users. This may include sample screen images, any GUI standards 
or product family style guides that are to be followed, screen layout 
constraints, standard buttons and functions (e.g., help) that will appear on 
every screen, keyboard shortcuts, error message display standards, and so on.  
Define the software components for which a user interface is needed. Details of 
the user interface design should be documented in a separate user interface 
specification.$>$

\section{Hardware Interfaces}
The Recipe Buddy client requires a modern web browser to access.

Please refer to the web browser's minimum specifications in order to get an idea of the required hardware.

\begin{itemize}
    \item \href{https://support.google.com/chrome/a/answer/7100626?hl=en}{Chrome browser system requirements}
    \item \href{https://en.wikipedia.org/wiki/Safari_(web_browser)#System_requirements}{Safari browser system requirements}
    \item \href{https://www.mozilla.org/en-US/firefox/92.0/system-requirements/}{Firefox System Requirements}
\end{itemize}

\section{Software Interfaces}
$<$Describe the connections between this product and other specific software 
components (name and version), including databases, operating systems, tools, 
libraries, and integrated commercial components. Identify the data items or 
messages coming into the system and going out and describe the purpose of each.  
Describe the services needed and the nature of communications. Refer to 
documents that describe detailed application programming interface protocols.  
Identify data that will be shared across software components. If the data 
sharing mechanism must be implemented in a specific way (for example, use of a 
global data area in a multitasking operating system), specify this as an 
implementation constraint.$>$

\section{Communications Interfaces}
The Recipe Buddy client interface runs in a web browser and communicates via the HTTP/HTTPs protocol. A high speed internet connection is required to use the full features of Recipe Buddy.

\chapter{System Features}
$<$This template illustrates organizing the functional requirements for the 
product by system features, the major services provided by the product. You may 
prefer to organize this section by use case, mode of operation, user class, 
object class, functional hierarchy, or combinations of these, whatever makes the 
most logical sense for your product.$>$

\section{System Feature 1}
$<$Don’t really say “System Feature 1.” State the feature name in just a few 
words.$>$

\subsection{Description and Priority}
$<$Provide a short description of the feature and indicate whether it is of 
High, Medium, or Low priority. You could also include specific priority 
component ratings, such as benefit, penalty, cost, and risk (each rated on a 
relative scale from a low of 1 to a high of 9).$>$

\subsection{Stimulus/Response Sequences}
$<$List the sequences of user actions and system responses that stimulate the 
behavior defined for this feature. These will correspond to the dialog elements 
associated with use cases.$>$

\subsection{Functional Requirements}
$<$Itemize the detailed functional requirements associated with this feature.  
These are the software capabilities that must be present in order for the user 
to carry out the services provided by the feature, or to execute the use case.  
Include how the product should respond to anticipated error conditions or 
invalid inputs. Requirements should be concise, complete, unambiguous, 
verifiable, and necessary. Use “TBD” as a placeholder to indicate when necessary 
information is not yet available.$>$

$<$Each requirement should be uniquely identified with a sequence number or a 
meaningful tag of some kind.$>$

REQ-1:	REQ-2:

\section{System Feature 2 (and so on)}


\chapter{Other Nonfunctional Requirements}

\section{Performance Requirements}
$<$If there are performance requirements for the product under various 
circumstances, state them here and explain their rationale, to help the 
developers understand the intent and make suitable design choices. Specify the 
timing relationships for real time systems. Make such requirements as specific 
as possible. You may need to state performance requirements for individual 
functional requirements or features.$>$

\section{Safety Requirements}
$<$Specify those requirements that are concerned with possible loss, damage, or 
harm that could result from the use of the product. Define any safeguards or 
actions that must be taken, as well as actions that must be prevented. Refer to 
any external policies or regulations that state safety issues that affect the 
product’s design or use. Define any safety certifications that must be 
satisfied.$>$

\section{Security Requirements}
$<$Specify any requirements regarding security or privacy issues surrounding use 
of the product or protection of the data used or created by the product. Define 
any user identity authentication requirements. Refer to any external policies or 
regulations containing security issues that affect the product. Define any 
security or privacy certifications that must be satisfied.$>$

\section{Software Quality Attributes}
$<$Specify any additional quality characteristics for the product that will be 
important to either the customers or the developers. Some to consider are: 
adaptability, availability, correctness, flexibility, interoperability, 
maintainability, portability, reliability, reusability, robustness, testability, 
and usability. Write these to be specific, quantitative, and verifiable when 
possible. At the least, clarify the relative preferences for various attributes, 
such as ease of use over ease of learning.$>$

\section{Business Rules}
$<$List any operating principles about the product, such as which individuals or 
roles can perform which functions under specific circumstances. These are not 
functional requirements in themselves, but they may imply certain functional 
requirements to enforce the rules.$>$


\chapter{Other Requirements}
$<$Define any other requirements not covered elsewhere in the SRS.\@ This might 
include database requirements, internationalization requirements, legal 
requirements, reuse objectives for the project, and so on. Add any new sections 
that are pertinent to the project.$>$

\section{Appendix A:\@ Glossary}
%see https://en.wikibooks.org/wiki/LaTeX/Glossary
$<$Define all the terms necessary to properly interpret the SRS, including 
acronyms and abbreviations. You may wish to build a separate glossary that spans 
multiple projects or the entire organization, and just include terms specific to 
a single project in each SRS.$>$

\section{Appendix B:\@ Analysis Models}
$<$Optionally, include any pertinent analysis models, such as data flow 
diagrams, class diagrams, state-transition diagrams, or entity-relationship 
diagrams.$>$

\section{Appendix C:\@ To Be Determined List}
$<$Collect a numbered list of the TBD (to be determined) references that remain 
in the SRS so they can be tracked to closure.$>$

\end{document}
